%! Author = thibaultchausson
%! Date = 28/11/2022

En analysant les performances de ces cinq métaheuristiques, il est notable que les algorithmes à population (comme l'algorithme génétique), la recherche taboue et le recuit simulé sont particulièrement efficaces pour ce type de problème.
Cette efficacité s'explique en grande partie par la facilité à modéliser le problème de façon similaire aux emplois du temps courants, mais aussi une exploration efficace de l'espace des solutions et d'obtenir rapidement de bons résultats.

Concernant l'approche par colonie de fourmies, bien que nous ayons adapté le problème des emplois du temps sous forme de graphe pour le rapprocher au problème du voyageur de commerce (où cet algorithme excelle), les améliorations apportées à la solution restent limitées, ne faisant pas de cette méthode la plus efficace pour notre cas.
Quant à l'essaim particulaire, nous n'avons pas réussi à trouver une modélisation adaptée à ce problème.

En résumé, pour la résolution de problèmes d'emploi du temps, les méthodes les plus efficaces s'avèrent être l'algorithme génétique, la recherche taboue et le recuit simulé.
Pour améliorer encore cette solution, nous pouvons créer avec l'algorithme génétique une très bonne solution et l'améliorer avec une recherche taboue ou un recuit simulé.