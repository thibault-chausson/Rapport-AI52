%! Author = thibaultchausson
%! Date = 11/12/2023

\subsection {Paramètres du problème}

Pour appliquer la méthode du recuit simulé, il nous faut premièrement en définir les paramètres.

Tout d'abord, la solution initiale est obtenue en générant aléatoirement une répartition des cours, dans les limites des contraintes fortes :


\begin{table}[!h]
    \centering
    \begin{tabular}{|c|c|c|c|c|c|c|c|c|c|c|c|}
        \hline
        \diagbox{Solution}{Cours} & 1  & 2 & 3 & 4 & 5  & 6 & 7 & 8 & 9  & 10  \\
        \hline
        $S_0$                    	   & J & Me & Ma & V & V & J  & L & V & V & Me        \\
        \hline
    \end{tabular}
    \caption{$S_0$ recuit simulé}\label{tab:s-0-recuit}
\end{table}


\begin{table}[!h]
    \centering
    \begin{tabular}{|c|c|c|c|c|c|}
        \hline
        Jours & L    & Ma    & Me   & J    & V    \\
        \hline
        $S_0$ & 7 & 3 & 2, 10 & 1,6 & 4, 5, 8, 9 \\
        \hline
    \end{tabular}
    \caption{$S_0$ initiale (jours)}\label{tab:s-0-recuit-jour}
\end{table}

Nous pouvons maintenant étudier la fitness de cette solution :

\begin{itemize}
	\item Classe 1 : 1 conflit, fitness égale à 1
	\item Classe 2 : 2 conflits, fitness égale à 0
	\item Classe 3 : 1 conflit, fitness égale à 1
	\item Classe 4 : 1 conflit, fitness égale à 1
	\item Classe 5 : 1 conflit, fitness égale à 1
\end{itemize}

Pour une fitness totale de 4

Pour des raisons pratiques, on limitera le recuit simulé à 5 itérations par palier, à un facteur de refroidissement $\lambda = 0.5$, à une température initiale $T_0 = 2$, et à un seuil de $0.3$, afin de n'effectuer que 15 opérations.

On définit une solution voisine comme toute solution respectant les contraintes fortes obtenue en déplaçant un cours à un jour adjacent.

On choisit la transformation à effectuer en choisissant un entier $c$ non nul dans l'intervalle $[-10;10]$

Un nombre négatif correspond à un déplacement du cours numéroté $|c|$ au jour précédent (on considère que le jour précédant le lundi est le vendredi), et inversement pour un nombre positif.

\subsection {Application}

{\setlength{\parindent}{0cm}

On commence l'algorithme avec $T=2$

On tire au hasard et on obtient $c = -1$

On recule donc le cours numéro 1 au jour précédent, qui est le Mercredi.

On obtient donc la solution suivante :


\begin{table}[!h]
    \centering
    \begin{tabular}{|c|c|c|c|c|c|}
        \hline
        Jours & L    & Ma    & Me   & J    & V    \\
        \hline
        $S_0$ & 7 & 3 & 1, 2, 10 & 6 & 4, 5, 8, 9 \\
        \hline
    \end{tabular}
\end{table}

Cette solution ayant une fitness de 5, le delta de fitness est de $\Delta f = 1$, et la nouvelle solution est immédiatement adoptée.

Itération suivante :

$c = 2$

Nouvelle solution : 

\begin{table}[!h]
    \centering
    \begin{tabular}{|c|c|c|c|c|c|}
        \hline
        Jours & L    & Ma    & Me   & J    & V    \\
        \hline
        $S_0$ & 7 & 3 & 1, 10 & 2, 6 & 4, 5, 8, 9 \\
        \hline
    \end{tabular}
\end{table}

$\Delta f = 1$

Adoption immédiate

Itération suivante :

$c = 9$

Nouvelle solution : 

\begin{table}[!h]
    \centering
    \begin{tabular}{|c|c|c|c|c|c|}
        \hline
        Jours & L    & Ma    & Me   & J    & V    \\
        \hline
        $S_0$ & 7, 9 & 3 & 1, 10 & 2, 6 & 4, 5, 8 \\
        \hline
    \end{tabular}
\end{table}

$\Delta f = 1$

Adoption immédiate

Itération suivante :

$c = -1$

Nouvelle solution : 

\begin{table}[!h]
    \centering
    \begin{tabular}{|c|c|c|c|c|c|}
        \hline
        Jours & L    & Ma    & Me   & J    & V    \\
        \hline
        $S_0$ & 7, 9 & 1, 3 & 10 & 2, 6 & 4, 5, 8 \\
        \hline
    \end{tabular}
\end{table}

$\Delta f = -1$

Probabilité d'acceptation : $0.6065306597126334$

Tirage de $x$ :
$x = 0.280299954665289$

Solution acceptée

Itération suivante :

$c = -6$

Nouvelle solution : 

\begin{table}[!h]
    \centering
    \begin{tabular}{|c|c|c|c|c|c|}
        \hline
        Jours & L    & Ma    & Me   & J    & V    \\
        \hline
        $S_0$ & 7 ,9 & 1 ,3 & 6, 10 & 2 & 4, 5, 8 \\
        \hline
    \end{tabular}
\end{table}

$\Delta f = 0$

Adoption immédiate

Après un cycle, on constate que la fitness est passée de 4 à 6

On réduit la température d'un facteur de $0.5$, et on obtient $T_1 = 1$, puis on recommmence un cycle, à la fin duquel on atteint la solution suivante :

\begin{table}[!h]
    \centering
    \begin{tabular}{|c|c|c|c|c|c|}
        \hline
        Jours & L    & Ma    & Me   & J    & V    \\
        \hline
        $S_0$ & 8, 9 & 3, 7, 3 & 1, 6 & 2, 5 & 4 \\
        \hline
    \end{tabular}
\end{table}

Pour une fitness de 7.

Arpès un dernier cycle à une température de $T_2 = 0.5$, on finit sur la solution suivante :


\begin{table}[!h]
    \centering
    \begin{tabular}{|c|c|c|c|c|c|}
        \hline
        Jours & L    & Ma    & Me   & J    & V    \\
        \hline
        $S_0$ & 8, 9 & 3, 7, 3 & 6 & 1, 2, 5 & 4 \\
        \hline
    \end{tabular}
\end{table}

Avec une fitness de 8. La température suivante, $T_3 = 0.25$, étant inférieure au seuil, l'algorithme s'arrête ici.

L'algorithme est donc efficace. D'autres tests nous ont permis de constater qu'abaisser le seuil à 0.2, amenant le nombre de cycles à 4, suffit quasi-systématiquement à atteindre la fitness maximale de 10.

}