%! Author = thibaultchausson
%! Date = 11/12/2023

Pour appliquer la méthode du recuit simulé, il nous faut tout d'abord en définir les paramètres.

Tout d'abord, la solution initiale est obtenue en générant aléatoirement une répartition des cours, dans les limites des contraintes fortes :


\begin{table}[!h]
    \centering
    \begin{tabular}{|c|c|c|c|c|c|c|c|c|c|c|c|}
        \hline
        \diagbox{Solution}{Cours} & 1  & 2 & 3 & 4 & 5  & 6 & 7 & 8 & 9  & 10  \\
        \hline
        $S_0$                    	   & J & Me & Ma & V & V & J  & L & V & V & Me        \\
        \hline
    \end{tabular}
    \caption{$S_0$ recuit simulé}\label{tab:s-0-recuit}
\end{table}


\begin{table}[!h]
    \centering
    \begin{tabular}{|c|c|c|c|c|c|}
        \hline
        Jours & L    & Ma    & Me   & J    & V    \\
        \hline
        $S_0$ & 7 & 3 & 2, 10 & 1,6 & 4, 5, 8, 9 \\
        \hline
    \end{tabular}
    \caption{$S_0$ initiale (jours)}\label{tab:s-0-recuit-jour}
\end{table}

Nous pouvons maintenant étudier la fitness de cette solution :

\begin{itemize}
	\item Classe 1 : 1 conflit, fitness égale à 1
	\item Classe 2 : 2 conflits, fitness égale à 0
	\item Classe 3 : 1 conflit, fitness égale à 1
	\item Classe 4 : 1 conflit, fitness égale à 1
	\item Classe 5 : 1 conflit, fitness égale à 1
\end{itemize}

Pour des raisons pratiques, on limitera le recuit simulé à 5 itérations par palier, à un facteur de refroidissement $\lambda = 0.5$, à une température initiale $T_0 = 2$, et à un seuil de $0.3$, afin de n'effectuer 15 opérations.

On définit une solution voisine comme toute solution respectant les contraintes fortes obtenue en déplaçant un cours à un jour adjacent.

On choisit la transformation à effectuer en choisissant un entier $x$ non nul dans l'intervalle $[-10;10]$

Un nombre négatif correspond à un déplacement du cours numéroté $|x|$ au jour précédent (on considère que le jour précédant le lundi est le vendredi), et inversement pour un nombre positif.




