%! Author = thibaultchausson
%! Date = 11/12/2023

%!TEX root = ../main.tex

\subsection{Planning}

L'objectif de ce projet est de présenter les métaheuristiques suivantes sur l'optimisation d'emploi du temps :
\begin{enumerate}
    \item Algorithme génétique
    \item Optimisation par colonies de fournies
    \item Optimisation par essaim particulaire
    \item Recuit simulé
    \item Recherche tabou
\end{enumerate}

Pour ce faire, nous allons définir le problème de la manière suivante.
Soient, $n \in \mathbb{N}^{*+}$, $s \in \llbracket 1 ,7 \rrbracket$, dans notre cas $s = 5$ qui sera le nombre de jours de la semaine, $c \in \llbracket 1 ,n \rrbracket$, le nombre de cours disponible, avec $n = 10 $.
Prenons la matrice creuse $M \in \mathbb{M}_{s,c}$, ou les $1$ représenter les cours dispensés ce jour :

\[
    M = \qquad \bordermatrix{~  & \tikzmark{harrowleft} Lundi & Mardi & Mercredi & Jeudi
    & Vendredi\tikzmark{harrowright}  \cr
    \tikzmark{varrowtop}
    1 : GE41 & 0 & 1 & 0 & 0 & 0 \cr
    2 : CC01 & 1 & 0 & 0 & 0 & 0 \cr
    3 : AI50 & 1 & 0 & 0 & 0 & 0 \cr
    4 : AP4A & 0 & 0 & 0 & 0 & 1 \cr
    5 : EI03 & 0 & 0 & 1 & 0 & 0 \cr
    6 : LJ00 & 0 & 0 & 0 & 1 & 0 \cr
    7 : SY41 & 0 & 0 & 0 & 1 & 0 \cr
    8 : SI40 & 0 & 0 & 0 & 0 & 1 \cr
    9 : IT42 & 0 & 0 & 1 & 0 & 0 \cr
    \tikzmark{varrowbottom} 10 : WE4A & 1 & 0 & 0 & 0 & 0 \cr
    }
\]
\tikz[overlay,remember picture] {
    \draw[->] ([yshift=3ex]harrowleft) -- ([yshift=3ex]harrowright)
    node[midway,above] {\scriptsize Jours de la semaine};
    \draw[->] ([yshift=1.5ex,xshift=-2ex]varrowtop) -- ([xshift=-2ex]varrowbottom)
    node[near end,left] {\scriptsize Cours};
}

Pour simplifier les différentes exécutions de ce problème, nous considérons qu'un cours dur une journée entière.
De plus, nous avons suffisamment de salle cours et professors pour dispenser l'ensemble de ces cours.

\subsection{Classes}

Nous considérons un ensemble de $d \in \mathbb{N}^{*+}$ classe constituées de $e \in \mathbb{N}^{*+}$ v\oe ux de cours, comme cela peu se faire à l'\gls{UTBM}.
Ici $d = 5$ et $e = 3$.
Représenté de la manière suivante :

\begin{center}
    \begin{tabular}{|c|c|}
        \hline
        Classe   & Cours                        \\
        \hline
        Classe 1 & 1 : GE41, 3 : AI50, 6 : LJ00 \\
        \hline
        Classe 2 & \ldots                       \\
        \hline
        Classe 3 & \ldots                       \\
        \hline
        Classe 4 & \ldots                       \\
        \hline
        Classe 5 & \ldots                       \\
        \hline
    \end{tabular}
\end{center}

Nous considérons que deux classes peuvent avoir le même cours, ainsi la direction aux formations pédagogiques affectera assez de professeurs.

\subsection{Contraintes}

Les contraintes fortes :
\begin{itemize}
    \item Tous les cours voulus par les classes dans l'emploi du temps
    \item Au moins $e = 3$ jours avec des cours dispensé, car les étudiants doivent participer à $e = 3$ cours
\end{itemize}

La contrainte faible :
\begin{itemize}
    \item Satisfaire les v\oe ux de tous les étudiants (classes)
\end{itemize}

\subsection{Fitness}

Nous définissions la $fitness$ comme suit : $p_i$ le score d'une classe, avec $p \in \llbracket 0 , 2 \rrbracket$ et $i \in \llbracket 1 , d \rrbracket$.
Si :
\begin{itemize}
    \item Aucun cours en même temps $p_i = 2$
    \item Deux cours en même temps $p_i = 1$
    \item Trois cours en même temps $p_i = 0$
\end{itemize}

Donc, nous avons :

\begin{equation}
    fitness = \sum_{i = 1}^{d} p_i\label{eq:fitness}
\end{equation}

Ainsi, nous avons $\max(fitness) = 2 \times d$, donc $\max(fitness) = 10$

\subsection{Une solution}

Prenons les v\oe ux des classes suivantes :

\begin{center}
    \begin{tabular}{|c|c|}
        \hline
        Classe   & Cours                         \\
        \hline
        Classe 1 & 1 : GE41, 3 : AI50, 6 : LJ00  \\
        \hline
        Classe 2 & 4 : AP4A, 5 : EI03, 9 : IT42  \\
        \hline
        Classe 3 & 3 : AI 50, 4 : AP4A, 8 : SI40 \\
        \hline
        Classe 4 & 2 : CC01, 4 : AP4A, 5 : EI03  \\
        \hline
        Classe 5 & 2 : CC01, 3 : AI50, 10 : WE4A \\
        \hline
    \end{tabular}
\end{center}

Le planning suivant :

\[
    M = \qquad \bordermatrix{~  & \tikzmark{harrowleft} Lundi & Mardi & Mercredi & Jeudi
    & Vendredi\tikzmark{harrowright}  \cr
    \tikzmark{varrowtop}
    1 : GE41 & 1 & 0 & 0 & 0 & 0 \cr
    2 : CC01 & 0 & 0 & 0 & 1 & 0 \cr
    3 : AI50 & 0 & 1 & 0 & 0 & 0 \cr
    4 : AP4A & 0 & 0 & 1 & 0 & 0 \cr
    5 : EI03 & 1 & 0 & 0 & 0 & 0 \cr
    6 : LJ00 & 0 & 0 & 1 & 0 & 0 \cr
    7 : SY41 & 0 & 0 & 0 & 0 & 1 \cr
    8 : SI40 & 0 & 0 & 0 & 1 & 0 \cr
    9 : IT42 & 0 & 1 & 0 & 0 & 0 \cr
    \tikzmark{varrowbottom} 10 : WE4A & 0 & 0 & 0 & 0 & 1 \cr
    }
\]
\tikz[overlay,remember picture] {
    \draw[->] ([yshift=3ex]harrowleft) -- ([yshift=3ex]harrowright)
    node[midway,above] {\scriptsize Jours de la semaine};
    \draw[->] ([yshift=1.5ex,xshift=-2ex]varrowtop) -- ([xshift=-2ex]varrowbottom)
    node[near end,left] {\scriptsize Cours};
}

Ainsi, nous avons la fitness suivante :

\begin{equation}
    fitness = \sum_{i = 1}^{d} p_i = 2 + 2 + 2 + 2 + 2 = 10 = \max(fitness)\label{eq:fitness_opti}
\end{equation}

Donc, ce planning optimise les v\oe ux des étudiants.

\subsection{Généralisation du problème}

%%%%%%%% Documentation : %%%%%%%%%
% https://perso.ensta-paris.fr/~kielbasi/tikzuml/var/files/doc/tikzumlmanual.pdf


Voici une représentation sous forme de diagramme UML de toutes les données que nous pouvons utiliser pour générer un calendrier.

\begin{center}
    \begin{figure}[!h]
        \begin{tikzpicture}
            \begin{umlpackage}{Data modeling}
                \umlclass[x=-3, y=-5]{Classroom}{
                    Id: UID \\
                    Number: string \\
                    Type: [CM, TP] \\
                    Seat: Int \\
                    Equipment: [Chemistry, Physics, \ldots]
                }{
                }
                \umlclass[x=1, y=0]{Session}{
                    Id : UID \\
                    Type: [CM, TD, TP] \\
                    Duration : Float\\
                    StartTime : DateTime
                }{
                }
                \umlclass[x=3, y=-5]{UV}{
                    Id: UID \\
                    Name: String \\
                    Pole: String \\
                }{
                }
                \umlclass[x=-5, y=0]{Teacher}{
                    Id: UID \\
                    Name: String \\
                    DayOff: String \\
                    Pole: String
                }{
                }
                \umlclass[x=3, y=-10]{Student}{
                    Id: UID \\
                    Name: String \\
                    UV : Array of UV \\
                    Pole: String \\
                    Level: String
                }{
                }
            \end{umlpackage}
            \umlassoc[geometry=--, arg1=, mult1=1, pos1=0.2,  arg2=, mult2=1..n, pos2=0.7]{Session}{Teacher}
            \umlassoc[geometry=|-|, arg1=, mult1=1, pos1=0.3,  arg2=, mult2=1..n, pos2=2]{Session}{Classroom}
            \umluniaggreg[geometry=|-, arg1=, mult1=1..n, pos1=0.15,  arg2=, mult2=1, pos2=0.4]{UV}{Session}
            \umlassoc[geometry=|-|, arg1=-enrolled List, mult1=0..n, pos1=0.4,  arg2=-UVs List, mult2=0..n, pos2=2.2]{UV}{Student}
        \end{tikzpicture}
        \caption{UML data} \label{fig:UML-data}
    \end{figure}
\end{center}

\newpage

