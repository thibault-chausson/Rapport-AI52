%%%%%%%% Documentation : %%%%%%%%%
% https://perso.ensta-paris.fr/~kielbasi/tikzuml/var/files/doc/tikzumlmanual.pdf


Voici une représentation sous forme de diagramme UML de toutes les données que nous pouvons utiliser pour générer un planning.

\begin{center}
    \begin{figure}[!h]
        \begin{tikzpicture}
            \begin{umlpackage}{Data modeling}
                \umlclass[x=-3, y=-5]{Classroom}{
                    Id: UID \\
                    Number: string \\
                    Type: [CM, TP] \\
                    Seat: Int \\
                    Equipment: [Chemistry, Physics, \ldots]
                }{
                }
                \umlclass[x=1, y=0]{Session}{
                    Id : UID \\
                    Type: [CM, TD, TP] \\
                    Duration : Float\\
                    StartTime : DateTime
                }{
                }
                \umlclass[x=3, y=-5]{UV}{
                    Id: UID \\
                    Name: String \\
                    Pole: String \\
                }{
                }
                \umlclass[x=-5, y=0]{Teacher}{
                    Id: UID \\
                    Name: String \\
                    DayOff: String \\
                    Pole: String
                }{
                }
                \umlclass[x=3, y=-10]{Student}{
                    Id: UID \\
                    Name: String \\
                    UV : Array of UV \\
                    Pole: String \\
                    Level: String
                }{
                }
            \end{umlpackage}
            \umlassoc[geometry=--, arg1=, mult1=1, pos1=0.2,  arg2=, mult2=1..n, pos2=0.7]{Session}{Teacher}
            \umlassoc[geometry=|-|, arg1=, mult1=1, pos1=0.3,  arg2=, mult2=1..n, pos2=2]{Session}{Classroom}
            \umluniaggreg[geometry=|-, arg1=, mult1=1..n, pos1=0.15,  arg2=, mult2=1, pos2=0.4]{UV}{Session}
            \umlassoc[geometry=|-|, arg1=-enrolled List, mult1=0..n, pos1=0.4,  arg2=-UVs List, mult2=0..n, pos2=2.2]{UV}{Student}
        \end{tikzpicture}
        \caption{UML data} \label{fig:UML-data}
    \end{figure}
\end{center}

\newpage