%! Author = thibaultchausson, nassimetallaoui
%! Date = 28/11/2022

%!TEX root = ../main.tex

Dans le cadre de nos études d'ingénieur en informatique, nous réalisons un projet sur la réalisation automatisée et optimale d'un emploi du temps pour l'\gls{UTBM}\cite{TimeTableScheduling}.
Pour ce faire, nous avons mis en place une stratégie de résolution basée sur les algorithmes génétiques\cite{burke1994genetic}, voulant explorer d'autres horizons, nous avons décidé de continuer cette étude en utilisant d'autres métaheuristiques, tels que le recuit simulé, la recherche taboue, \ldots\

Ainsi, nous réaliserons :
\begin{itemize}
    \item Modéliser mathématiquement le problème (critère à optimiser, contraintes, variables,..)
    \item Définir les paramètres de la métaheuristique (exemple codage, croisement et mutation pour AG : Température, facteur de décroissance,\ldots\ pour RS : Voisinage, taille de liste taboue, mouvements tabous,\ldots\ pour RT : nombre de fourmis, facteur de visibilité, de trace, coefficient d'évaporation,\ldots\ pour ACO : inertie de particule, facteurs d'attraction, positions, vitesses,\ldots pour PSO)
    \item Exécuter quelques itérations de la méthode sur une instance de petite taille.
    Il ne s’agit pas forcément de faire le programme pour chaque méthode, mais il s’agit plutôt de montrer.
\end{itemize}

